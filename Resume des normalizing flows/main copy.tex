\documentclass{article}

% adjusting margin
% \usepackage[a4paper, margin=1in]{geometry}

% for the appendices
\usepackage[toc, page]{appendix}




% adjusting indentation and paragraph spacing
\setlength\parindent{0pt}
\setlength{\parskip}{0.4em}

% formatting for abstract
\newenvironment{vplace}[1][1]
  {\par\vspace*{\stretch{#1}}}
  {\vspace*{\stretch{1}}\par}
\renewcommand{\abstractname}{{\scshape \Large \textmd{Abstract}}\\}

% using natbib for references
\usepackage[numbers]{natbib}
\usepackage{url}

\usepackage{yfonts}

\usepackage{nccmath}
\newenvironment{mpmatrix}{\begin{medsize}\begin{pmatrix}}%
    {\end{pmatrix}\end{medsize}}%
    
\newcommand\myrule[1]{\multicolumn{1}{| l}{#1}}
\newcommand\myrules[1]{\multicolumn{2}{| l}{#1}{| l}{#1}}


\usepackage{blkarray}
\usepackage{stackengine}
\usepackage{bbding}
\usepackage{calc}
\usepackage{graphicx}
\usepackage{amsmath}
\usepackage[export]{adjustbox}
\usepackage{amsthm}
\usepackage{amssymb}
\usepackage{makeidx}
\usepackage{relsize}
\usepackage{multirow}
\usepackage[ruled,vlined]{algorithm2e}
\usepackage{bm}
\usepackage{sidecap}
\usepackage{ upgreek }
\usepackage{xcolor} %% note colors
\usepackage{graphicx}
\usepackage{multicol}
%\usepackage{pgfbaselayers}
\usepackage{times}
\usepackage{helvet}
\usepackage{stmaryrd}
\usepackage{palatino}
\usepackage[utf8]{inputenc}
\usepackage[breaklinks]{hyperref}
\usepackage{aurical}
\usepackage[T1]{fontenc}
\usepackage{mathtools}
\newcommand\SmallMatrix[1]{{%
\tiny\arraycolsep=0.3\arraycolsep\ensuremath{\begin{pmatrix}#1\end{pmatrix}}}}
\renewcommand{\Pr}{\mbox{P}}
\newcommand{\e}{\mbox{e}}
\newcommand{\dx}{\,\mbox{d}x}

%%%%% COMMANDS
%\newcommand{\x}{\mathbf{x}}
%\newcommand{\R}{\mathbb{R}}
%\newcommand{\p}{\mathbf{p}}
%\newcommand{\vp}{\mathbf{v}_\mathbf{p}}
%\newcommand{\q}{\mathbf{q}}

% formatting and numbering theorems etc.
\newtheorem{lemma}{Lemma}[section]
\newtheorem{proposition}[lemma]{Proposition}
\newtheorem{axiom}[lemma]{Axiom}
\newtheorem{theorem}[lemma]{Theorem}
\newtheorem{corollary}[lemma]{Corollary}
\newtheorem{conjecture}[lemma]{Conjecture}

\theoremstyle{definition}
\newtheorem{definition}[lemma]{Definition}


\theoremstyle{definition}
\newtheorem{examp}[lemma]{Example}

\theoremstyle{remark}
\newtheorem*{remark}{Remark}

%\theoremstyle{remark}
%\newtheorem*{claim}{Claim}
%\theoremstyle{remark}
%\newtheorem*{proofofclaim}{Proof of claim}

\setcounter{MaxMatrixCols}{20}

% Peiran's custom commands
\usepackage{enumitem}
\newcommand{\htwonormalised}{\begin{pmatrix} + & + \\ + & - \end{pmatrix}}
\newcommand{\GammaQuotient}{\Gamma/\langle u \rangle}
\newcommand{\GammaRQuotient}{\Gamma_{\mbox{\small \textnormal{right}}}/\langle -I \rangle}
\DeclareMathOperator{\Aut}{Aut}
\DeclareMathOperator{\Sym}{Sym}
\DeclareMathOperator{\Ker}{Ker}
\DeclareMathOperator{\Image}{Im}
\DeclareMathOperator{\LT}{LT}
\DeclareMathOperator{\LM}{LM}
\DeclareMathOperator{\LC}{LC}
\DeclareMathOperator{\lcm}{lcm}
\DeclareMathOperator{\multideg}{multideg}
\DeclareMathOperator{\Criterion}{Criterion}
\DeclareMathOperator{\st}{s.t. }
\DeclareMathOperator{\trace}{trace}
\DeclareMathOperator{\argmin}{argmin}
\DeclareMathOperator{\cut}{cut}

%\newcommand{P}{P}
%\newcommand{Q}{\mathbf{Q}}
%\newcommand{H}{H}
%\newcommand{I}{\mathbf{I}}
\makeatletter
\renewcommand*\env@matrix[1][c]{\hskip -\arraycolsep
  \let\@ifnextchar\new@ifnextchar
  \array{*\c@MaxMatrixCols #1}}
\makeatother


%for matlab code
\usepackage{listings}
\usepackage{color} %red, green, blue, yellow, cyan, magenta, black, white
\definecolor{mygreen}{RGB}{28,172,0} % color values Red, Green, Blue
\definecolor{mylilas}{RGB}{170,55,241}

\lstset{language=Matlab,%
    %basicstyle=\color{red},
    breaklines=true,%
    morekeywords={matlab2tikz},
    keywordstyle=\color{blue},%
    morekeywords=[2]{1}, keywordstyle=[2]{\color{black}},
    identifierstyle=\color{black},%
    stringstyle=\color{mylilas},
    commentstyle=\color{mygreen},%
    showstringspaces=false,%without this there will be a symbol in the places where there is a space
    numbers=left,%
    numberstyle={\tiny \color{black}},% size of the numbers
    numbersep=9pt, % this defines how far the numbers are from the text
    emph=[1]{for,end,break},emphstyle=[1]\color{red}, %some words to emphasise
    %emph=[2]{word1,word2}, emphstyle=[2]{style},    
}

% metadata




\begin{document}
\textbf{Rapha\"el Pellegrin, Gael Ancel, David Assaraf - 23/11/20\\
\begin{center}
Normalizing flows: an introduction and review of current methods \end{center}}
\medskip

\section{Introduction}

Normalizing flows are a family of generative models. It is a transfomration of a simple probability distribution into a more complicated one via a series of diffeomorphism. We can evaluate the density of a sample from the complex distribution by looking at the density of its pre-image in the simple distribution, multiplied by the appropriate change in volume induced by the transformations: this is given by the aboslute values of the determinant of the Jacobians of each transformation.

Standard applications of normalizing flows include density estimation and prior construction, but th eone that interests us in is oulier detection.

Let $Z \in \mathbb{R}^D$ be a random variable with probability density function (pdf):
$$p_Z:  \mathbb{R}^D \rightarrow \mathbb{R}$$ 
Then, for a diffeomorphism $g$, we can let $Y:=g(Z)$.

We then have:

\begin{align} F_Y(y)&= P(Y\leq y)  \\
&=P(g(Z)\leq y)\\
&=P(Z \leq g^{-1}(y))\\
&=F_Z(g^{-1}(y)) \end{align}

Thus:

\begin{align} p_Y(y)=p_Z(g^{-1}(y))|det Df(y)| \end{align}

where $f:=g^{-1}$. 

\begin{definition} $p_Y(y)$ is called the pushforward of the density $p_Z(z)$. $Z$ is called the base distribution, and the movement from $Z$ to $Y$ is called the generative direction. The inverse of $g$, $f$, moves in the normalizing direction. This explains the name normalizing direction.
\end{definition}

In practice, we choose the diffeomorphism with a neural network (Papamakarios et al.)

\section{Type of flows}

There a different types of flows: the kinds discussed in the paper are elementwise flows, linear flows, planar and radial flows, coupling and autoregressive flows, residual flows, infinitesimal (continuous flows).

Element wise flows use a bijection $h: \mathbb{R} \rightarrow \mathbb{R}$ to construct the bijection $g: \mathbb{R}^D \rightarrow \mathbb{R}^D$:

$$ g((x_1, \dots, x_D)^T) \mapsto (h(x_1), \dots, h(x_D))^T $$

Linear flows correspond to a function $g: \mathbb{R}^D \rightarrow \mathbb{R}^D$ such that:

$$ g(x)=Ax+b$$
where $A \in \mathbb{R}^{D\times D}$ and $b \in \mathbb{R}^{D}$.

\subsection{Coupling flows}

The most used flows are coupling and autoregressive flows.

\section{COmputational trade-offs}

We need to be able to sample from $p_Z(z)$, we meed to be able to e 







\end{document}

